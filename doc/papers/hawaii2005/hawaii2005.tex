\documentclass[11pt, conference, draftcls, letterpaper]{IEEEtran}

\usepackage[dvips]{graphicx}

\begin{document}

\title{Hybrib Parallel Finite Difference Time Domain Simulation of
  Nanoscale Optical Phenomenon}
\author{\authorblockN{M. C. Hughes} 
\authorblockA{Electrical and Computer Engineering\\
University of Victoria\\
Victoria, British Columbia, Canada\\
Email: mhughe@uvic.ca
}}

\maketitle

\begin{abstract}
  
\end{abstract}


\section{Introduction}
The dicovery of enhanced optical transmission by Ebessen
et. al. \cite{ebessen1998} in 1998 has lead to the 

\section{Parallel Methods}

\section{Metaprogramming}

\section{Optical Simulation}
NNN has reported that a laser focused on one array of nano-holes can
couple into surface plasmon modes, which travel along the surface of
the metal to a second array of holes, where the surface plasmon modes
are coupled back into propagating modes \cite{}. 

In order to explore this effect with FDTD, it is necessary to have
an extremly large grid, due to the size of the arrays and the space
between them. This problem cannot be analysed used periodic boundary
conditions, although it does posses one plane of symmetry. 

\section{Results}

\section{Conclusion}


\bibliographystyle{IEEEtran}
\bibliography{IEEEabrv,../biblio/phred}

\end{document}
